%\chapter{Abstract}
\thispagestyle{plain}
\begin{center}
    \Large{
    \textbf{Examples of Dynamical Reduction Using Symmetry \\for Simple Mechanical Systems with Virtual Holonomic Constraints}
    }
        
    \vspace{0.4cm}
    \large{\scshape
    Emrys Halbertsma
    }\\
    \normalsize{
        Department of Physics \& Astronomy,\\
        University of Waterloo
    
        \vspace{0.4cm}
        Supervised by\\}
    \large{\scshape
    Dr. Christopher Nielsen
    }\\
    \normalsize{
        Department of Electrical \& Computer Engineering,\\
        University of Waterloo
    }
    \vspace{1.0cm}
    
\begin{minipage}{0.85\textwidth}
\normalsize
    \textbf{Abstract:} A mechanical system can be made to appear holonomically constrained by actively enforcing a constraint through control. These so-called Virtual Holonomic Constraints (VHCs) have useful applications in robotics, such as maintaining set distances between agents or synchronizing the motions of joints. 
% A mechanical system subject to such VHCs, the system is driven towards a submanifold of its configuration manifold, but its dynamics are not necessarily Euler-Lagrange on the constraint manifold.
This project considers some examples of simple, 2\tss{nd}-degree underactuated mechanical systems with 1 degree of symmetry. Using Matlab, the mechanical systems are modelled and analyzed. The tangent spaces of their constraint manifolds are computed, along with the system's constrained dynamics, and then decomposed into an Ehresmann-like connection of Vertical and Horizontal Vector fields. The dynamical reduction technique can prove useful in simplifying the dynamics of robotic applications requiring the enforcement of a Virtual Holonomic Constraints.
\vspace{5mm}

\small\textbf{Key terms:} Virtual Holonomic Constraint, symmetry, geometric control, dynamical reduction.
\end{minipage}
\end{center}