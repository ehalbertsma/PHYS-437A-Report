%\chapter{Abstract}
\thispagestyle{plain}
\begin{center}
    \Large{
    \textbf{Examples of Dynamical Reduction Using Symmetry \\for Simple Mechanical Systems with Virtual Holonomic Constraints}
    }
        
    \vspace{0.4cm}
    \large{\scshape
    Emrys Halbertsma
    }\\
    \normalsize{
        Department of Physics \& Astronomy,\\
        University of Waterloo
    
        \vspace{0.4cm}
        Supervised by\\}
    \large{\scshape
    Dr. Christopher Nielsen
    }\\
    \normalsize{
        Department of Electrical \& Computer Engineering,\\
        University of Waterloo
    }
    \vspace{1.0cm}
    
\begin{minipage}{0.85\textwidth}
\normalsize
    \textbf{Abstract:} A mechanical system can be made to appear holonomically constrained by actively enforcing a constraint through control. These so-called Virtual Holonomic Constraints (VHCs) have useful applications in robotics, such as maintaining set distances between agents or synchronizing the motions of joints. 

% A mechanical system subject to such VHCs, the system is driven towards a submanifold of its configuration manifold, but its dynamics are not necessarily Euler-Lagrange on the constraint manifold.
\todo[inline]{Discuss appraoch and utility of the dynamical reduction and summarize results}
This project considers some examples of simple, 2\tss{nd}-degree underactuated mechanical systems with 1 degree of symmetry. Using Matlab, the mechanical systems are modelled and analyzed.

\vspace{5mm}
\textbf{Key terms:} Virtual Holonomic Constraint, symmetry, geometric control, dynamical reduction.
\end{minipage}
\end{center}