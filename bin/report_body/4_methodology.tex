\documentclass[main.tex]{subfiles}
\begin{document}
\chapter{Methodology}
% ------------------------
% ------------------------
\section{Simple Mechanical Systems with Symmetries}
In this section, we discuss a number of examples of systems of interest. These are simple mechanical systems to which we can apply dynamical reduction.
\subsection{Example: Rolling Disk}\todo{this doesn't actually have a symmetry, just keeping it in as a placeholder}
This example is retrieved from \cite[9]{bullo2019geometric}.

\textbf{Description} Consider a round disk whose circular edge is constrained to the surface of a flat plane. The coordinates of the system are $q=\pmqty{x&y&\theta&\phi}^T$, and the system is actuated by $\tau=\pmqty{\tau_\theta&\tau_\phi}^T$.

\textbf{Constraints} The disk rolls along the plane with no slip, just like a regular wheel. The no-slip condition can be expressed as a set of holonomic constraints:
\begin{align}
    \pmqty{\ddot{x}\\\ddot{y}}-\rho\dot{\phi}\pmqty{\cos\theta\\\sin\theta}=\pmqty{\zmat{2}{1}}
\end{align}

\textbf{Lagrangian} The disk's potential is constant. Its kinetic energy is defined as,
\begin{align}
    KE = \frac{1}{2}\dot q^T \mathbb{G}(q) \dot q, \textrm{ where  }\mathbb{G}(q)=
\mqty(\dmat[0]{m,m,J_{roll},J_{spin}}).
\end{align}
Then its Lagrangian is given by
\begin{align}
    \Lagr(q,\dot q)=\frac{1}{2}m(\dot x^2+\dot y^2)+\frac{1}{2}J_{roll}\dot\theta^2+\frac{1}{2}J_{spin}\dot{\phi}^2.
\end{align}

\subsection{Example: Robotic Leg} 
This example is retrieved from \cite[293]{bullo2019geometric}.%

\textbf{Description} Consider a massive body which rotates around a fixed point in space. A leg extends from the body, which can rotate relative to the body and change its length.

\textbf{Constraints}

\textbf{Lagrangian}

\textbf{Symmetries}

\subsection{Example: Spherical Pendulum in Gravity}
This example is retrieved from \cite[295]{bullo2019geometric}.%

\textbf{Description} A point of mass $m$ is attached to a massless arm of length $\ell$ in a gravitational field with potential $P=-g\hat{s}_3$. The configuration manifold is therefore $Q=\mathbb{S}^2(\ell)$, the kinetic energy metric is $\mbbg=$\todo{example 4.99}.

\textbf{Constraints} The system is constrained to the sphere $\mathbb{S}^2(\ell)$ since the pendulum length is fixed.

\textbf{Lagrangian}

\textbf{Symmetries}



\end{document}