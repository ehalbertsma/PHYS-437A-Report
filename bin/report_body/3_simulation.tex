\documentclass[main.tex]{subfiles}
\begin{document}
\section{Simulation}
\subsection{Overview}
Matlab was chosen for its extensive documentation, and for its popularity in industrial control systems design. This project made use of Matlabs's Symbolic Toolbox and Control System Toolbox.

Simulations of the systems described in Section \ref{sec:examples} were conducted in Matlab. As mentioned, some previous work had been done by a former student, so this was used as a starting point. In particular, the computation of Christoffel symbols and parametrized constraint dynamics methods are built on this previous work. 

The program accepts a configuration file for a simple mechanical system definition, its coordinate relationships, and a set of VHCs. The system dynamics are then computed and modeled, and then simulated over a time interval. 

\subsection{Program Architecture}
% \begin{lstlisting}
% | constrained`_system_dynamics.m
% ├── data
% │   ├── robot_arm_lewis
% │   │   ├── C_function.m
% │   │   ├── D_function.m
% │   │   ├── G_function.m
% │   │   ├── dhq_function.m
% │   │   ├── dhqdt_function.m
% │   │   ├── h_function.m
% │   │   └── plots
% │   ├── robot_arm_offset
% │   │   ├── Ac_function.m
% │   │   ├── C_function.m
% │   │   ├── D_function.m
% │   │   ├── G_function.m
% │   │   ├── bc_function.m
% │   │   ├── dhq_function.m
% │   │   ├── dhqdt_function.m
% │   │   ├── h_function.m
% │   │   └── plots
% │   └── spherical_pendulum_mccarthy
% │       ├── C_function.m
% │       ├── D_function.m
% │       ├── G_function.m
% │       ├── dhq_function.m
% │       ├── dhqdt_function.m
% │       ├── h_function.m
% │       └── plots
% ├── example-systems
% │   ├── robot_arm_lewis.m
% │   ├── robot_arm_offset.m
% │   └── spherical_pendulum_mccarthy.m
% ├── generate_template.m
% ├── main.m
% ├── new_system_template.m
% ├── plot_sandbox.m
% ├── plot_system.m
% ├── simulate_system.m
% ├── system_differential_equation.m
% └── system_dynamics.m
% \end{lstlisting}
%\todo{include a diagram of program architecture}
% \todo{update this with constraint dynamics}
\begin{enumerate}
    \item The user starts at \verb|new_system_template.m|. 
    When prompted, the user enters the system name, and a new directory and configuration file are generated.
    \item Next, the user navigates to the new configuration file to enter the system parameters.
    \item The system can then be imported using \verb|system_dynamics.m|, which converts the configured parameters into a model of the mechanical system. The relevant Lagrangian, inertia, coriolis, and actuator functions are saved in the system's subdirectory.
    This script makes use of Matlab's Symbolic Toolbox. 
    \item From there, \verb|simulate_system.m| sets up the differential equation that we wish to solve. The script calls on a series of sub-functions to model the system term by term at each timestep of the simulation. 
    \item Finally, \verb|plot_system.m| outputs visualizations of the system dynamics. These are saved in a subfolder with the same name as the system.
\end{enumerate}


\subsection{Simulating a Smooth Simple Control System}
The program uses Matlab's \verb|ode45| solver, an implementation of the Range-Kutta algorithm with varying time steps\cite{senan2007brief}.

The solver iterates through a specified interval and timestep to solve Equation \ref{control-equation} corresponding to the instantaneous $D(q),C(\qd,q),P(q),$and $B$ matrices. The system's behaviour is then corrected using the control signal $\tau$ to drive the system towards the constraint set defined by the VHCs.

Other required parameters include:
\begin{enumerate}[(a)]
    \item PD controller constants $K_P,K_D$,
    \item Numerical values for segment lengths, masses, $B$-matrix, and gravitational constant
    \item A vector \verb|q0| of initial positions and velocities of the coordinates.
\end{enumerate}

The program outputs and saves a series of plots visualizing the system's behaviour.
\subsection{Version Control}
The project is stored on Dr. Nielsen's Gitlab instance. A new branch titled \verb|fall2022-dev| contains the newly developed code.
\end{document}